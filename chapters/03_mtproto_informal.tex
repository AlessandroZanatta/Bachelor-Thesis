
In this chapter, we will give a brief overview of the MTProto2.0 protocol. A more in-depth and formal description can be found on the official web page \cite{Telegram-MTProto2.0}.

First of all, MTProto2.0 is a \textit{suite of protocols} used to enable secure communication between a Telegram client and a Telegram server over an insecure network. MTProto2.0 can be decomposed in the three following protocols:

\begin{description}[style=nextline]
    \item[Authorization] used to obtain a secret authorization key shared only with the server;
    \item[Secret-chat] used to obtain a secret key shared between two clients. This is then used to exchange end-to-end messages between two clients, with the server that basically acts as a forwarder;
    \item[Rekeying] used to achieve Perfect Forward Secrecy, allows obtaining a new end-to-end encryption key.
\end{description}

Additionally, the cloud chats encryption schema is used to securely exchange messages between clients and servers (who have shared an authorization key).

An overview on these protocols will be given in \cref{sec:auth-prot,sec:cloud-chat,sec:secret-chat,sec:rekeying}.

\section{Authorization protocol}
\label{sec:auth-prot}

%% Authorization protocol %%
\begin{figure}[!t]
\setlength{\instdist}{4cm}
\setmscoptions
\begin{msc}{}
\setmscscale{.8} 

\declinst{client}{}{Client}
\declinst{server}{}{Server}

\action*{Generates nonces $n_{c}, n_{k}$}{client}
\action*{\parbox{4.5cm}{\centering 
    Knows keys $\mbox{sk}^{(1)}, \dots, \mbox{sk}^{(n)}$\\
    Generates $n_s, g, p$\\
    Generates proof-of-work primes $q, r$
}}{server}
\nextlevel[6]

\mess{$n_{c}$}{client}{server}
\nextlevel[2]
\mess{$n_{c}, n_{s}, q \cdot r, \mbox{fp}^{(1)}, \dots, \mbox{fp}^{(n)}$}{server}{client}

\nextlevel
\action*{\parbox{4cm}{\centering
    Chooses $\mbox{pk}^{(i)}$ matching\\
    $\mbox{fp}^{(i)}$ for some $i$\\
    Factorizes $q \cdot r$\\
    $C_1 := q \cdot r, q, r, n_c, n_s, n_k$
}}{client}

\nextlevel[7]
\mess{$n_c, n_s, q, r, \mbox{fp}^{(i)}, \{\mbox{sha1}\left(C_1\right), C_1\}_{\mbox{pk}^{(i)}}$}{client}{server}
\nextlevel

\action*{$\left(k, iv\right) := \mbox{kdf}\left(n_s, n_k\right)$}{client}
\action*{\parbox{4.5cm}{\centering
    $s \in \Z_p$\\
    $g_s := g^s \mod{p}$\\
    $key, iv := \mbox{kdf}\left(n_s, n_k\right)$\\
    $S_1 := n_c, n_s, g, p, g_s, t_1$
}}{server}

\nextlevel[6]
\mess{$n_c, n_s, \left\{\mbox{sha1}\left(S_1\right), S_1\right\}_{key, iv}$}{server}{client}
\nextlevel

\action*{\parbox{4.5cm}{\centering
$c \in \Z_p$\\
$g_c := g^c \mod{p}$\\
Checks $g, p$\\
$k_{CS} := g_s^c \mod{p}$\\
$C_2 := n_c, n_s, rID, g_c$
}}{client}
\nextlevel[7]

\mess{$n_c, n_s, \left\{\mbox{sha1}\left(C_2\right), C_2\right\}_{k, iv}$}{client}{server}
\nextlevel

\action*{\parbox{4.5cm}{\centering
$k_{CS} := g_c^s \mod{p}$
}}{server}
\nextlevel[3]

\mess{$n_c, n_s, \mbox{hash}\left(n_k, k_{CS}\right)$}{server}{client}


\end{msc}
\centering
\caption{MTProto2.0 Authorization protocol}
\label{fig:authorization-protocol}
\end{figure}


The first time a Telegram client C runs the application, it must negotiate an \textbf{authorization key} with the Telegram server S. The authorization protocol is used to this end. Once the client and the server have shared an authorization key, they will use it to encrypt (almost) every future communication between them. A client might also have several keys (e.g. on multiple devices or if reinstalling the application), some of which might be locked (e.g. if the device is lost). The authorization protocol is based on the Diffie-Hellman key exchange protocol \cite{DH-protocol}.


A successful protocol run consists of three rounds, which are represented schematically in \cref{fig:authorization-protocol}:
\begin{description}
    \item[Round 1] In the first round messages are in plaintext. In particular, the client and the server exchange two nonces ($n_c$ and $n_s$). The pair $\left<n_c, n_s\right>$ identifies a session of the authorization protocol. These nonces are sent in every consequent message of the current run of the protocol, both in plaintext and encrypted form.
    \begin{enumerate}
        \item{In the first message, the client sends his fresh nonce $n_c$ to the server;}
        \item{The server answers with the client nonce $n_c$, the server fresh nonce $n_s$, a challenge $q \cdot r \leq 2^{63}$ (which are two primes used as a measure to prevent denial of service, as the client needs to spend resources on factorizing $q \cdot r$ before the server has to commit (more) resources\footnote{Notice that this might not be true as this is vulnerable to a lookup table approach (e.g. using \href{http://factordb.com}{factordb.com}).}) and a list of public RSA key fingerprints (calculated as the lower 64-bits of the SHA1 of the server public keys).}
    \end{enumerate}

    \item[Round 2] The client decomposes $q\cdot r$ in $\left<q, r\right>$, retrieves the public key of the server $pk^{\left(i\right)}$. The client then generates a nonce $n_k$ of 256 bits. This nonce $n_k$ is supposed to be secret. The pair $\left<n_s, n_k\right>$ is used, by both server and client, to derive, through a derivation function $\mbox{kdf}$, a symmetric encryption key $k$ and an initialization vector $iv$, which will be used in subsequent exchanges.
    \begin{enumerate}
        \setcounter{enumi}{2}
        \item{The client asymmetrically encrypts both $\left<q\cdot r, q, r, n_c, n_s, n_k\right>$ and its SHA1 hash with $pk^{(i)}$ and sends it along with $\left<n_c, n_s, q, r, fp^{(i)}\right>$ to the server. A rather complex padding schema is used;}
        \item{The server generates his Diffie-Hellman ephemeral key $s$ of 2048-bits, chooses $g, p$ and computes $g_s = g^s \mod{p}$. Finally, it symmetrically encrypts both $\left<n_c, n_s, g, p, g_s, t_1\right>$ and its SHA1 hash and sends it along with $\left<n_c, n_s\right>$.}
    \end{enumerate}

    Notice that the client is supposed to check that:
    \begin{itemize}
        \item{$p$ is a safe 2048-bit prime, where safe means that both $p$ and $\frac{p-1}{2}$ are prime and $2^{2047} < p < 2^{2048}$;}
        \item{$g$ is a generator for $\frac{p-1}{2}$.}
    \end{itemize}

    \item[Round 3] In the last round, the client generates his own Diffie-Hellman ephemeral key and shares it with the server.
    \begin{enumerate}
        \setcounter{enumi}{4}
        \item{The client generates his ephemeral key $c$ of 2048-bits and computes $g_c = g^c \mod{p}$. Then, it symmetrically encrypts both $\left<n_s, n_s, retryID, g_c\right>$ and its SHA1 hash and sends them along with $\left<n_c, n_s\right>$. The $retryID$ starts at zero at the time of the first attempt, otherwise is based on the values from the last failed attempt;}
        \item{The server is now able to compute the authorization key as $k_{CS} = g_c ^ s \mod{p}$. Assuming server checks pass, S sends an acknowledgment for the new key: $\left<n_c, n_s, \mbox{hash}\left(k_{CS}\right)\right>$.}
    \end{enumerate}

\end{description}

\subsection{Implementation notes}
TODO (?)








\section{Cloud-chat encryption schema}
\label{sec:cloud-chat}

%% Cloud-chat encryption schema %%
\begin{figure}[t]
    \centering
    \includegraphics{cloud-chats}
    \caption{MTProto2.0 Cloud-chat protocol.\\Representation inspired by the Telegram official one.}
    \label{fig:cloud-chat-protocol}
    \end{figure}

Telegram uses the schema in \cref{fig:cloud-chat-protocol} to encrypt every message exchanged between the client and the server after an authorization key has been established using the authorization protocol in \cref{sec:auth-prot}.

A message key \textbf{msg\_key} of 128 bits is calculated as the middle 128 bits of the SHA256 of the entire message prepended by 32 bytes of the authorization key. The message itself contains a 64-bit salt, a 64-bit session id, the payload\footnote{The payload contains the time of the message, its length, a sequence number. Receiver should check these pieces of information, after decryption.} and a variable size padding of 12-1024 bytes.
The authorization key \textbf{auth\_key}, combined with the message key \textbf{msg\_key}, is used to derive a key and an initialization vector, which are used to encrypt the entire message using AES in IGE mode.

\subsection{IGE mode}
Infinite Garble Extension (in short, IGE) is a block cipher mode, lesser-known than others like ECB, CBC, OFB, CTR, CFB, GCM, CCM.
IGE can be defined with the following formula:

\begin{equation}
c_i = f_K(m_i \oplus c_{i-1}) \oplus m_{i-1}
\end{equation}

where $f_K$ stands for the encrypting function (like AES) with key $K$
and $i$ goes from 1 to $n$ $–$ the number of plaintext blocks. Two initialization vectors are also needed. \Cref{fig:IGE} summarizes how the encryption in IGE mode works.

\begin{figure}[t]
    \centering
    \includegraphics{IGE}
    \caption{Encryption in IGE mode.}
    \label{fig:IGE}
\end{figure}

One of the main properties of IGE mode is that it makes sure that if a ciphertext block is changed, then every subsequent block following it will not decrypt correctly.

As pointed out by \cite{Telegram-AFAQ-IGE}, the Telegram developers team is aware of the vulnerability of this mode to blockwise-adaptive Chosen Plaintext Attack (CPA)\cite{IGE-CPA}, but they claim that MTProto is not affected.




\section{Secret-chat protocol}
\label{sec:secret-chat}

%% Secret-chat protocol %%
\begin{figure}[!t]
\setlength{\instdist}{3cm}
\setmscoptions
\begin{msc}{}
\setmscscale{.8} 

\declinst{alice}{}{Alice}
\declinst{server}{}{Server}
\declinst{bob}{}{Bob}


\action*{\parbox{4.5cm}{\centering
    Knows $k_{AS}, k_{BS}$
}}{server}

\action*{\parbox{4.5cm}{\centering
    Knows $k_{AS}$
}}{alice}

\action*{\parbox{4.5cm}{\centering
    Knows $k_{BS}$
}}{bob}

\nextlevel[2]
\action*{\parbox{4.5cm}{\centering
    Gets $g, p$\\
    Generates $sid$\\
    $a \in \Z_p$\\
    $g_a := g^a \mod{p}$\\
    $M_1 := g, p, sid, g_a$
}}{alice}

\nextlevel[7]

%% TODO: Telegram webpage does not seem to specify HOW this is encrypted!!
%% UPDATE: It actually (kinda) does: it's encrypted as a cloud-chat message! For simplicity, keep this notation and explain its meaning in the written part.
\mess{$\left\{M_1\right\}_{k_{AS}}$}{alice}{server}
\nextlevel
\mess{$\left\{M_1\right\}_{k_{BS}}$}{server}{bob}

\nextlevel
\action*{\parbox{4.5cm}{\centering
    $b \in \Z_p$\\
    $g_b := g^b \mod{p}$\\
    $k_{AB} := g_a ^ b \mod{p}$\\
    $M_2 := sid, g_b, \mbox{fpk}\left(k_{AB}\right)$
}}{bob}

\nextlevel[6]
\mess{$\left\{M_2\right\}_{k_{BS}}$}{bob}{server}
\nextlevel
\mess{$\left\{M_2\right\}_{k_{AS}}$}{server}{alice}

\nextlevel
\action*{\parbox{5cm}{\centering
    $k_{AB} := g_b ^ a \mod{p}$\\
    Generates $payload$\\
    $mk := \mbox{sha256}\left(k_{AB}, payload\right)$\\
    $key, iv = \mbox{kdf}\left(k_{AB}, mk\right)$\\
    $C := \left\{payload\right\}_{key, iv}$
    $msg := \mbox{fpk}\left(k_{AB}\right), mk, C$
}}{alice}

\nextlevel[8]
\mess{$\left\{msg\right\}_{k_{AS}}$}{alice}{server}
\nextlevel
\mess{$\left\{msg\right\}_{k_{BS}}$}{server}{bob}

\end{msc}

\centering
\caption{MTProto2.0 Secret-chat protocol}
\label{fig:secret-chat-protocol}
\end{figure}

Telegram secret chats deal with end-to-end encryption between two clients (as usual, let us call the clients Alice and Bob, or A and B for brevity). After both clients have shared an authorization key with the Telegram server S, they can decide to engage themselves in a run of the secret chat protocol, allowing them to share, using a Diffie-Hellman key exchange, a shared secret. Notice that the server acts as a forwarder: every message sent from a client A to another client B is sent to the server, encrypted as a cloud chat message shown in \cref{sec:cloud-chat} with the authorization key of A; the server then decrypts the message and encrypts it as a cloud chat message using the authorization key of B and finally sends it to B. The writing $\left\{M\right\}_{k_{AS}}$ in \cref{fig:secret-chat-protocol} expresses that the message $M$ has been encrypted with the authorization key $k_{AS}$ using the cloud chat encryption schema.




\section{Rekeying protocol}
\label{sec:rekeying}