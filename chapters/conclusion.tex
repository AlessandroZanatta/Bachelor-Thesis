We have shown the two models for cryptographic protocol verification and we have also described the symbolic model. We have shown how Tamarin prover and Proverif can be used to model a simple protocol in \cref{sec:simple_protocol}. Then, we have presented the MTProto2.0 protocol suite and its formalization using Tamarin prover. We have proved its security in our fully automatic model, and we have re-discovered that the protocol is vulnerable to a theoretical \uks{} (UKS) attack. This attack allows a malicious client B, with the help of another client E, to convince client A that she has shared a key with E, while, instead, she has shared it with B. Finally, we have compared both Tamarin prover and Proverif implementations and found that they have comparable efficiency. However, Tamarin prover excels on the expressiveness, especially considering the powerful built-in theories (such as \DiHe{}) available to the user.

Overall, we were able to reproduce the same results obtained with Proverif using Tamarin prover, with some simplifications and restrictions.