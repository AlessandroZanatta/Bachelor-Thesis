\def\thudbabelopt{english}
\PassOptionsToPackage{usenames}{xcolor}
\PassOptionsToPackage{dvipsnames}{xcolor}
\documentclass[target=bach,aauheader=]{thud}

%% --- Information about the thesis --- %%
\course{Internet of Things, Big Data and Web}
\title{Comparison of tools for the formal verification of MTProto 2.0}
\author{Alessandro Zanatta}
\supervisor{Prof.\ Marino Miculan}
\cosupervisor{Prof.\ Nicola Vitacolonna} % TODO: Is Vitacolonna considered a supervisor?
%% Other available fields: \reviewer, \tutor, \chair, \date (anno accademico, calcolato in automatico), \rights

%% --- Suggested packages ---
%% pdfx: per generare il PDF/A per l'archiviazione. Necessario solo per la versione finale
\usepackage[a-1b]{pdfx}
%% hyperref: Regola le impostazioni della creazione del PDF... più tante altre cose. Ricordarsi di usare l'opzione pdfa.
\usepackage[pdfa]{hyperref}
%% tocbibind: Inserisce nell'indice anche la lista delle figure, la bibliografia, ecc.

%% --- Stili di pagina disponibili (comando \pagestyle) ---
%% sfbig (predefinito): Apertura delle parti e dei capitoli col numero grande; titoli delle parti e dei capitoli e intestazioni di pagina in sans serif.
%% big: Come "sfbig", solo serif.
%% plain: Apertura delle parti e dei capitoli tradizionali di LaTeX; intestazioni di pagina come "big".

%% --- Other packages --- %%
\usepackage{msc} % Protocol graphical representation
\usepackage{mathtools}
\usepackage{amssymb}
\usepackage{enumitem}
\usepackage[justification=centering]{caption} % Used to have always centered captions
\usepackage{cleveref} % better references
\usepackage{graphicx} % for images
\usepackage{textgreek} % greek letters out of math mode
\usepackage{caption} % captions for equations (outside of floats, in general)
\usepackage{listings}
\usepackage[super]{nth}
\usepackage{multicol}
\usepackage[T1]{fontenc}
\usepackage{courier} % Font for listings

\newcommand\keywordstyle[1]{{\footnotesize\color{MidnightBlue}\bfseries\mathversion{bold}#1}}

\lstset{
  numbers=left,
  xleftmargin=\parindent,
  basicstyle=\small\linespread{1}\ttfamily,
  breaklines=true,
  captionpos=b,
  aboveskip=10pt,
  belowskip=10pt,
  literate=%
  {∃}{{\keywordstyle{$\exists$}}}1
  {∀}{{\keywordstyle{$\forall$}}}1
  {∧}{{\keywordstyle{$\land$}}}1
  {∨}{{\keywordstyle{$\lor$}}}1
  {¬}{{\keywordstyle{$\neg$}}}1
  {==>}{{\keywordstyle{$\Longrightarrow$}}}1
}
\graphicspath{ {./images/} } % images path

\DeclareCaptionType{equ}[][]

%% --- Commands --- %%
\newcommand\setmscoptions{
  % \setlength{\levelheight}{1.5 \levelheight}%
  % \setlength{\instwidth}{3cm}
  \setmsckeyword{}
  \drawframe{no}
  \centering
}


% Multiline comments
\newcommand{\comment}[1]{}

\newcommand*{\Z}{\mathbb{Z}}
\newcommand*{\Q}{\mathbb{Q}}

%% Taken from https://hal.inria.fr/file/index/docid/955869/filename/sapic.tex
\newcommand{\msrewrite}[1]{\mathrel{-\hspace{-2pt}[#1]\hspace{-4pt}\to}}
\newcommand{\emptyrule}{\ensuremath{[]}\xspace}
\newcommand{\msr}[3]{\ensuremath{#1 \msrewrite{#2} #3}}
%% -------------- %%

\newcommand{\msrnolabel}[2]{\ensuremath{#1 \rightarrow #2}}
\newcommand{\msrsetminus}{\ensuremath{\setminus^\#}}
\newcommand{\msrcap}{\ensuremath{\cap^\#}}
\newcommand{\msrcup}{\ensuremath{\cup^\#}}
\newcommand{\msrin}{\ensuremath{\in^\#}}
\newcommand{\msrsubseteq}{\ensuremath{\subseteq^\#}}
\newcommand{\lin}[1]{\ensuremath{lin\left(#1\right)}}
\newcommand{\pers}[1]{\ensuremath{pers\left(#1\right)}}

\newcommand{\fact}[2]{\ensuremath{\mbox{!#1}\left(#2\right)}}
\newcommand*{\myexp}{\hat{\mkern6mu}}

% pi-calculus
\newcommand{\pic}{\textpi-calculus }
\newcommand{\picnospace}{\textpi-calculus}
\newcommand{\apicpc}[2]{\ensuremath{P\ |\ Q}}
\newcommand{\apicin}[4]{\mbox{in}\left(#1, #2: #3\right); #4}
\newcommand{\apicout}[3]{\mbox{out}\left(#1, #2\right); #3}
\newcommand{\apicrep}[1]{!#1}
\newcommand{\apicnew}[3]{\ensuremath{\mbox{new}\ #1: #2; #3}}
\newcommand{\apiclet}[5]{\ensuremath{\mbox{let}\ #1: #2\ = #3\ \mbox{in}\ #4\ \mbox{else}\ #5}}
\newcommand{\apicif}[3]{\ensuremath{\mbox{if}\ #1\ \mbox{then}\ #2\ \mbox{else}\ #3}}
\newcommand{\apicevent}[2]{\ensuremath{\mbox{event}\ \mbox{#1}\left(#2\right)}}

% MTProto encryption symbols/functions
\newcommand{\func}[2]{\ensuremath{\mbox{#1}\left(#2\right)}}
\newcommand{\enc}[2]{\ensuremath{\left\{#1\right\}_{#2}}}
\newcommand{\sha}[2]{\ensuremath{\func{sha#1}{#2}}}
\newcommand{\kdf}[1]{\ensuremath{\func{kdf}{#1}}}
\newcommand{\fpk}[1]{\ensuremath{\func{fpk}{#1}}}
\newcommand{\hash}[1]{\ensuremath{\func{hash}{#1}}}
\newcommand{\modexp}[3]{\ensuremath{#1^#2 \mod{#3}}}
\newcommand{\key}[1]{\ensuremath{k_{#1}}}
\newcommand{\newkey}[1]{\ensuremath{k'_{#1}}}
\newcommand{\group}[1]{\ensuremath{\Z_{#1}}}

% spacing in math multiline mode
\setlength{\jot}{1pt}

%% -------------------------------------------------------------------------------- %%
%% Languages listings                                                               %%
%% -------------------------------------------------------------------------------- %%
\lstdefinelanguage{tamarin}
{
  keywordstyle=\color{MidnightBlue}\bfseries,
  keywordstyle=[2]\itshape,
  keywordstyle=[3]\color{Green}\bfseries,
  keywordstyle=[4]\color{RedViolet}\bfseries,
  alsoletter={^,==>,|,&,.},
  keywords={let, in, Out, In, K, KU, Fr, ^, senc, sdec, adec, aenc, ~, All, Ex, not, ., &, |, ==>, diff},
  keywords=[2]{},
  keywords=[3]{},
  keywords=[4]{rule, lemma},
  sensitive=true,
  morecomment=[l]{//},
  morecomment=[n][\color{OliveGreen}\itshape]{/*}{*/},
  morestring=[b]',
  stringstyle=\color{BrickRed},
}

\lstdefinelanguage{verifpal}
{
  keywordstyle=\color{MidnightBlue}\bfseries,
  keywordstyle=[2]\itshape,
  keywordstyle=[3]\color{Green}\bfseries,
  keywordstyle=[4]\color{Orange}\bfseries,
  alsoletter={->, ?},
  keywords={principal, attacker, queries, ->, ?},
  keywords=[2]{Client, Server, Alice, Bob},
  keywords=[3]{leaks, phase},
  keywords=[4]{active, passive},
  sensitive=true,
  morecomment=[l]{//},
  morestring=[b]",
}

\lstdefinelanguage{proverif}
{
  keywordstyle=\color{MidnightBlue}\bfseries,
  keywordstyle=[2]\itshape,
  keywordstyle=[3]\color{Green}\bfseries,
  keywords={table, query, insert, leaks, phase, get, out, event, in, ., ;, attacker, |, new, if, else, then, !, let},
  alsoletter={., ;, |, !},
  keywords=[2]{Client, Server},
  keywords=[3]{},
  sensitive=true,
  morecomment=[l]{//},
  morecomment=[n][\color{OliveGreen}\itshape]{(*}{*)},
  morestring=[b]",
}

%% -------------------------------------------------------------------------------- %%
%% Document start                                                                   %%
%% -------------------------------------------------------------------------------- %%
\begin{document}
\maketitle

%% Dedica (opzionale)
\begin{dedication}
  Al mio cane,\par per avermi ascoltato mentre ripassavo le lezioni.
\end{dedication}

%% Ringraziamenti (opzionali)
\acknowledgements
Ringraziamenti vari qua

%% Sommario (opzionale)
\abstract
Un bell'abstract va qua!

%% Indice
\tableofcontents

%% Lista delle tabelle (se presenti)
%\listoftables

%% Lista delle figure (se presenti)
%\listoffigures

%% Corpo principale del documento
\mainmatter

%% Capitolo
\chapter{Introduction}
\input{./chapters/01_introduction}

\chapter{Symbolic and Computational model}
\input{./chapters/02_symbolic_computational}

\chapter{MTProto2.0 protocol description}

In this chapter a brief overview of the MTProto2.0 protocol will be given. A more in-depth and formal description can be found on the official web page \cite{Telegram-MTProto2.0}.

First of all, MTProto2.0 is a \textit{suite of protocols} used to enable a secure communication between a client and an MTProto server over an insecure network. MTProto2.0 can be seen as composed by the following protocols:

\begin{itemize}
    \item{Authorization - used to obtain a secret authorization key shared only with the server}
    \item{Cloud-chat}
    \item{Secret-chat}
    \item{Rekeying}
\end{itemize}

An overview on these protocols will be given in the next sections.

\section{Authorization protocol}

%% Authorization protocol %%
\begin{figure}[htb!]
\centering
\caption{MTProto2.0 Authorization protocol}
\setmscoptions
\begin{msc}{}
\setmscscale{.8} 

\declinst{client}{}{Client}
\declinst{server}{}{Server}

\action*{Generates nonces $n_{c}, n_{k}$}{client}
\action*{\parbox{4.5cm}{\centering 
    Knows keys $\mbox{sk}^{(1)}, \dots, \mbox{sk}^{(n)}$\\
    Generates $n_s, g, p$\\
    Generates proof-of-work primes $q, r$
}}{server}
\nextlevel[6]

\mess{$n_{c}$}{client}{server}
\nextlevel[2]
\mess{$n_{c}, n_{s}, q \cdot r, \mbox{fp}^{(1)}, \dots, \mbox{fp}^{(n)}$}{server}{client}

\nextlevel
\action*{\parbox{4cm}{\centering
    Chooses $\mbox{pk}^{(i)}$ matching\\
    $\mbox{fp}^{(i)}$ for some $i$\\
    Factorises $q \cdot r$\\
    $C_1 := q \cdot r, q, r, n_c, n_s, n_k$
}}{client}

\nextlevel[7]
\mess{$n_c, n_s, q, r, \mbox{fp}^{(i)}, \{\mbox{sha1}\left(C_1\right), C_1\}_{\mbox{pk}^{(i)}}$}{client}{server}
\nextlevel

\action*{$\left(k, iv\right) := \mbox(kdf)\left(n_s, n_k\right)$}{client}
\action*{\parbox{4.5cm}{\centering
    $a \in \Z_p$\\
    $g_a := g^a \mod{p}$\\
    $k, iv := \mbox{kdf}\left(n_s, n_k\right)$\\
    $S_1 := n_c, n_s, g, p, g_a, t_1$
}}{server}

\nextlevel[6]
\mess{$n_c, n_s, \left\{\mbox{sha1}\left(S_1\right), S_1\right\}_{k, iv}$}{server}{client}
\nextlevel

\action*{\parbox{4.5cm}{\centering
$b \in \Z_p$\\
$g_b := g^b \mod{p}$\\
Checks $g, p, g_a, g_b$\\
$k_{AS} := g_a^b \mod{p}$\\
$C_2 := n_c, n_s, rID, g_b$
}}{client}
\nextlevel[7]

\mess{$n_c, n_s, \left\{\mbox{sha1}\left(C_2\right), C_2\right\}_{k, iv}$}{client}{server}
\nextlevel

\action*{\parbox{4.5cm}{\centering
$k_{AS} := g_b^a \mod{p}$
}}{server}
\nextlevel[3]

\mess{$n_c, n_s, \mbox{hash}\left(n_k, k_{AS}\right)$}{server}{client}


\end{msc}
\end{figure}








\section{Cloud-chat protocol}

%% Cloud-chat protocol %%
\begin{figure}[htb!]
\centering
\caption{MTProto2.0 Cloud-chat protocol}
\setmscoptions
\begin{msc}{}

\declinst{client}{}{Client}
\declinst{server}{}{Server}

\action*{\parbox{4cm}{\centering
    Knows $k_{AS}$\\
    Generates $salt, sid,$\\$msg, pad$\\
    $M := \mbox{salt}, \mbox{sid}, \mbox{msg}, \mbox{pad}$\\
    $msg\_key := \mbox{sha256}\left(M\right)$
}}{client}
\action*{Knows auth\_key}{server}
\nextlevel[6]

\mess{}{client}{server}


\end{msc}
\end{figure}




\chapter{Tools comparison}

\section{Proverif}
Let us start with a brief overview of Proverif internal reasoning. For more information, please refer to \cite{SymbolicComputationalBlanchet, SymbolicVerificationBlanchet, ProverifManual}.

\subsection{High level view}
Proverif protocols and security properties are based on an extended version of the \pic (the \textit{applied} \pic). The tool also allows the user to define constructors, destructors and equations\footnote{Destructors are basically used to de-construct some previously constructed term (e.g. decryption of an encrypted ciphertext), while equations represent term equality of some sort (e.g. commutativity of multiplication).}, which form the cryptographic primitives. The protocol is then automatically translated to a set of Horn-clauses. Using this abstract representation of the protocol (based on Horn-clauses), the Proverif verifier uses a resolution algorithm on such clauses that allows for verification of security properties \cite{SymbolicComputationalBlanchet}.
A graphical representation of the whole process is given in \cref{fig:proverif-verification-method}.

It is important to note that Proverif is not complete. This means that it may find false attacks. Moreover, it may not terminate, but it has been proven to be precise and efficient enough in practice by many case studies (the following is a non-exhaustive list of examples \cite{10.1145/1266977.1266978, ABADI20053, hal-01575923, DBLP:journals/corr/abs-2012-03141}).

We will now proceed with an overview of \pic and Horn-clauses.

\begin{figure}[t]
    \includegraphics{proverif-verification-method}
    \centering
    \caption{Proverif verification method.\\Inspired by a representation from Bruno Blanchet \cite{SymbolicComputationalBlanchet}.}
    \label{fig:proverif-verification-method}
\end{figure}

\subsection{\pic and applied \pic}

The \pic \cite{pi-calculus-book} is a (minimal) programming that models system communicating on channels. It belongs to the \textit{process calculi} family, which is generally used to model concurrent systems. As Proverif uses the \textit{applied} \pic (which is an extension of standard \picnospace), we are going to briefly present its syntax in the rest of this section.

The following description of applied \pic references articles \cite{applied-pi-calculus-private-auth, applied-pi-calculus-abadi-1, applied-pi-calculus-abadi-2}. Please refer to these resources for further information and a more formal or in-depth description. For brevity, we only define main features of applied \pic in \cref{subsub:syntax-apic}.

\subsubsection{Overview of the syntax of the applied \pic}
\label{subsub:syntax-apic}

A \textit{signature \textSigma} is composed by a finite number of functions symbols, each with its own integer arity. Given such signature, together with an infinite set of names and an infinite set of variables, the set of \textbf{terms} is defined by the grammar:

\begin{equation}
\label{eq:apic-terms}
\begin{aligned}
    U, V &::=\\
    &a, b, \dots\\
    &x, y, \dots\\
    &f\left(U_1, \dots, U_l\right)
\end{aligned}
\qquad
\begin{aligned}
    \mbox{term}&\mbox{s}\\
    &\mbox{name}\\
    &\mbox{variable}\\
    &\mbox{constructor application}
\end{aligned}
\end{equation}

where $f \in \Sigma$ and $l$ matches the arity of $f$. Next, we define a grammar for processes, which is shown in \cref{eq:apic-processes}. As pointed to by Microsoft researchers, this grammar is very similar to the \pic \cite{applied-pi-calculus-private-auth}. We will omit defining differences from standard \pic as we have not formally defined \pic either.

\begin{equation}
\label{eq:apic-processes}
\begin{aligned}
    P, Q &::= \\
    &0 \\
    &\apicout{N}{M}{P} \\
    &\apicin{N}{x}{T}{P} \\
    &\apicpc{P}{Q}\\
    &\apicrep{P}\\
    &\apicnew{a}{T}{P}\\
    &\apicif{M}{P}{Q}
\end{aligned}
\qquad
\begin{aligned}
    \mbox{proc}&\mbox{esses} \\
    &\mbox{null process}\\
    &\mbox{output to channel N of message M}\\
    &\mbox{input from channel N of message M with sort T}\\
    &\mbox{parallel composition}\\
    &\mbox{replication}\\
    &\mbox{fresh value of sort T}\\
    &\mbox{conditional}
\end{aligned}
\end{equation}

The null process $0$ does nothing;
$\apicout{N}{M}{P}$ ($\apicin{N}{x}{T}{P}$) outputs (gets) the message M (of sort $x$) into (from) channel N and then continues with process $P$; Notice that getting a message from a channel is a blocking operation;
$\apicpc{P}{Q}$ is the parallel composition of $P$ and $Q$;
The process $\apicrep{P}$ effectively behaves as an infinite number of copies of $P$ running in parallel (\textit{unbounded} replication);
$\apicnew{a}{T}{P}$ creates a new fresh value of sort $T$, before proceeding with process $P$;
$\apicif{M}{P}{Q}$ if a standard conditional.

\comment{
$\apiclet{x}{T}{D}{P}{Q}$ is used to apply destructors or assign some term $D$ to a variable $x$ (of sort $T$);
} 



\subsection{Horn-clauses}







\section{Tamarin-prover}
In this section we will see an overview of Tamarin foundations and internal reasoning.
For a more in-depth description and further information, see the Tamarin foundations paper \cite{TamarinFoundations} or the extended foundations paper \cite{TamarinFoundationsExtended}.

\subsection{High level view}
First of all, let us examine an high level picture of Tamarin.

The security property model of Tamarin is based on labelled multiset rewriting rules to specify protocols and adversary capabilities, a guarded fragment\footnote{Only a few examples of formulas respecting the guarded fragment of first order logic used by Tamarin will be given in \cref{sub:guarded-formulas}. See \cite{FragmentFirstOrderLogicPaper} for a rigorous definition from a mathematical point of view.} of first order logic to specify security properties\footnote{Security properties in Tamarin will also be referred to as \textit{lemmas}.} and functions and equational theories to model the algebraic properties of cryptographic protocols \cite{TamarinFoundations}.

Given the rewriting rules, security properties and equational theories, Tamarin uses a novel constraint-solving algorithm which tries to validate or falsify lemmas.

In other words, Tamarin allows to specify a labelled transition system that induces a set of traces and offers verification of such traces using a guarded fragment of first-order logic to specify ``good" traces. Tamarin then tries to prove the negation of the specified ``good" traces.

Tamarin also offers builtin equational theories \cite{TamarinProverManual}. A brief overview will be given in \cref{sub:Builtin-equational-theories}.

\subsection{Terminology}
As reported earlier, multiset rewriting rules are used to specify adversary capabilities and protocols. More precisely, a \textit{set} of \textit{labelled} multiset rewriting rules are used.

The ingredients of this multiset rewriting system are the following: 

\begin{description}[style=nextline]
    \item[Terms] which can be essentially thought of as messages. Terms can be of three different sorts. The more general sort is the \textit{msg} sort, which has two incomparable subsorts \textit{fresh} and \textit{pub} for fresh and public names, respectively;
    \item[Facts] which model information in the protocol. Facts have an arity, can be linear or persistent and are composed by terms. Linear facts model resources that can be consumed once, while persistent facts can be consumed an arbitrary number of times (and are prefixed by an exclamation mark). By convention, facts always start with a capital letter;
    \item[Special facts] Four facts are reserved and are used to model the freshness of a message $t$ ($\mbox{\textbf{Fr}}\left(t\right)$), a message $t$ coming from the public channel ($\mbox{\textbf{In}}\left(t\right)$), a message $t$ to be output to the public channel ($\mbox{\textbf{Out}}\left(t\right)$) and knowledge of a certain message $t$ from the attacker ($\mbox{\textbf{K}}\left(t\right)$);
    \item[State of the system] The state of the system is represented using a \textit{multiset} of facts;
    \item[Transition rules] A multiset of transition rules defines the possible transitions from one state to another one. Transitions are denoted with the following syntax
    \begin{equation}
        L \msrewrite{A} R
    \end{equation}
    where $L, A$ and $R$ are multisets of facts, respectively called \textbf{premises}, \textbf{actions} and \textbf{conclusions}.
    \item[Trace] A trace is a sequence $\left<A_1, \dots, A_n\right>$ of sets of ground facts denoting the sequence of actions that happened during a protocol's execution.
\end{description}


\subsection{Transition rules}
\label{sub:Transition-rules}
Let us examine an informal description of transitions.

\begin{itemize}
    \item{Let $S$ be the current state of the system}
    \item{Let $\msrnolabel{L}{R}$ be a transition rule. Note that this is a \textit{multiset rewriting rule} without a \textit{label};}
    \item{Let $\msrnolabel{l}{r}$ be a ground instance of the rule (i.e. no variables are present in the multisets);}
    \item{If we apply $\msrnolabel{l}{r}$ (assuming $l \msrsubseteq S$) to our state $S$ we reach a new state, defined by the following equation:
    \begin{equation}
        S' = S \msrsetminus l \msrcup r
    \end{equation}
    We use $\msrsetminus$, $\msrcup$ and $\msrsubseteq$ to define difference, union and subset over multisets, respectively. We can also consider the difference between linear and persistent facts and define $\lin{l}$ ($\pers{l}$) as linear facts (persistent facts) in $l$. Assuming that $\lin{l} \msrsubseteq S$ and $\pers{l} \msrsubseteq S$, then the equation becomes the following:
    \begin{equation}
        S' = S \msrsetminus \lin{l} \msrcup r
    \end{equation}
    It should be fairly clear from the equation why persistent facts can be consumed any number of times: they are never removed from the state of the system. This can be useful in scenarios in which we want to model persistent knowledge (e.g. the establishment of an encryption key $k$ may be expressed by a persistent fact $\fact{Key}{k}$).
    }
    \item{When we use labelled multiset rewriting rules, such as $\msr{l}{a}{r}$, we also add facts from $a$ to the \textit{trace} of the execution.}
\end{itemize}

\Cref{eq:builtin-tamarin-msr-rules} shows multiset rewriting rules that are always defined by Tamarin. It is not hard to see that these equations are used to model the Dolev-Yao attacker: the first rule allows the adversary to send a message $x$ he knows to someone, while the second one allows him to learn a message $x$ sent by someone.

\begin{equation}
\label{eq:builtin-tamarin-msr-rules}
\begin{gathered}
    \msr{!KU\left(x\right)}{K\left(x\right)}{In\left(x\right)}\\
    \msrnolabel{Out\left(x\right)}{\ !KD(x)}
\end{gathered}
\end{equation}


\subsection{Builtin equational theories}
\label{sub:Builtin-equational-theories}
A brief list of Tamarin built-ins is given below. Only the builtin theories considered relevant and those used in the analysis will be described here. The full list is available in the Tamarin manual. \cite{TamarinProverManual}.

\begin{description}[style=nextline]
    \item[hashing] defines a perfect hash function \textbf{h/1}\footnote{The writing \textbf{f/x} indicates that the function \textbf{f} has arity \textbf{x}.};
    \item[asymmetric-encryption] models a public key encryption scheme. It defines the following symbols:
    
    \begin{itemize}
        \item{\textbf{aenc/2}, used to model the encryption of a message with a public key}
        \item{\textbf{adec/2}, used to model the decryption of an encrypted message with a private key}
        \item{\textbf{pk/1}, used to derive a public key from a private key}
    \end{itemize}

    Functions are related by the equation \textbf{adec(aenc(msg, pk(sk)), sk) = msg};

    \item[diffie-hellman] models Diffie-Hellman groups. It defines the following symbols:
    
    \begin{itemize}
        \item{\textbf{inv/1}, models the inverse of an element}
        \item{\textbf{1/0}, models the neutral element}
        \item{\textbf{$\myexp$} and \textbf{*} symbols, models exponentiation and multiplication respectively}
    \end{itemize}

    The equational theory for this builtin is actually quite complex. For the sake of completeness, these are the related equations:
    \begin{equation}
    \begin{aligned}
        &x \myexp y \myexp z = x \myexp \left(y * z\right) \\
        &x ^ 1 = x \\
        &x * y = y * x \\
        &\left(x * y\right) * z = x * \left(y * z\right) \\
        &x * 1 = x \\
        &x * inv\left(x\right) = 1
    \end{aligned}
    \end{equation}
    % TODO: parla ancora un po' di quanto sia figo Tamarin che ha DH builtin!
\end{description}

\subsection{Guarded formulas}
\label{sub:guarded-formulas}

As reported earlier, Tamarin uses a guarded fragment of first order logic to specify security properties and $-$ in general $-$ traces. All formulas must be guarded, which essentially means that all variables quantified \textbf{must} appear in facts. \Cref{eq:guarded-formulas} shows two main formulas that respect the guarded fragment. Most $-$ if not every $-$ security property can be expressed using these formulas:

\begin{equation}
\label{eq:guarded-formulas}
\begin{gathered}
    \forall \overline{x}. F\left(\overline{z}\right) @i \Rightarrow \psi \\
    \exists \overline{x}. F\left(\overline{z}\right) @i \land \psi
\end{gathered}
\end{equation}

where $F$ is a fact, $\psi$ is guarded and $\overline{x}$ and $\overline{z}$ are vectors of terms such that $\overline{x} \subseteq \overline{z} \cup i$.
The variable $i$ is a timepoint, which annotates that the fact $F$ occurred at time $i$ (timepoints all belong to $\Q$)\footnote{Notice that timepoints are very useful to define post-compromise security properties (i.e. leaking an ephemeral key \textbf{after} honest parties have used it) and Perfect Forward Secrecy.} \cite{TamarinTeachingSlides}.

\section{Comparison of Proverif and Tamarin}
\label{section:proverif-vs-tamarin}

\comment{
    Proverif is NOT complete (see SymbolicComputationalBlanchet page 10), while Tamarin actually is (probably see foundations paper).
}

\chapter{Formalisation in Tamarin-Prover}
In this section we will dive into the implementation of the MTProto2.0 protocol created in Tamarin prover. The full code is available on github \cite{MTProto2-Tamarin}. Please refer to the README instructions for the code structure and for how to run the code.

This formalisation is based on the paper \cite{MTProto2-Proverif} and the Proverif analysis \cite{MTProto2-Proverif-impl} of MTProto2.0 by M. Miculan and N. Vitacolonna.

We will proceed analyzing the implementation of every single protocol and schema described in \cref{sec:mtproto2-informal}.

\section{Authorization protocol}
\label{sec:auth-prot-formalisation}
\subsection{Exchanges formalisation}
Let us describe, round by round, how the protocol was formalised. See \cref{fig:formalisation-authorization-protocol} for the updated schematic of the protocol.

%% Authorization protocol %%
\begin{figure}[!t]
  \setlength{\instdist}{4cm}
  \setmscoptions
  \begin{msc}{}
    \setmscscale{.8}

    \declinst{client}{}{Client}
    \declinst{server}{}{Server}

    \action*{Generates nonces $n_{c}, n_{k}$}{client}
    \action*{\parbox{4.5cm}{\centering
        Knows keys $\mbox{sk}^{(1)}, \dots, \mbox{sk}^{(n)}$\\
        Generates $n_s$
      }}{server}
    \nextlevel[4]

    \mess{$n_{c}$}{client}{server}
    \nextlevel[2]
    \mess{$n_{c}, n_{s}, \mbox{fp}^{(x)}$}{server}{client}

    \nextlevel
    \action*{\parbox{4cm}{\centering
        Gets $\mbox{pk}^{(x)}$ using $\mbox{fp}^{(x)}$\\
        $C_1 := n_c, n_s, n_k$
      }}{client}

    \nextlevel[5]
    \mess{$n_c, n_s, \mbox{fp}^{(x)}, \enc{C_1}{\mbox{pk}^{(i)}}$}{client}{server}
    \nextlevel

    \action*{$\left(k, iv\right) := \func{genKey}{n_s, n_k}$}{client}
    \action*{\parbox{4.5cm}{\centering
        $s \in \group{p}$\\
        $g_s := g^s$\\
        $key, iv := \func{genKey}{n_s, n_k}$\\
        $S_1 := n_c, n_s, g_s$
      }}{server}

    \nextlevel[6]
    \mess{$n_c, n_s, \enc{S_1}{key, iv}$}{server}{client}
    \nextlevel

    \action*{\parbox{4.5cm}{\centering
        $c \in \group{p}$\\
        $g_c := g^c$\\
        $\key{CS} := g_s^c$\\
        $C_2 := n_c, n_s, g_c$
      }}{client}
    \nextlevel[7]

    \mess{$n_c, n_s, \enc{C_2}{k, iv}$}{client}{server}
    \nextlevel

    \action*{\parbox{4.5cm}{\centering
        $\key{CS} := g_c^s$
      }}{server}
    \nextlevel[3]

    \mess{$n_c, n_s, \hash{n_k, \key{CS}}$}{server}{client}
  \end{msc}
  \centering
  \caption{MTProto2.0 Authorization protocol (simplified)}
  \label{fig:formalisation-authorization-protocol}
\end{figure}

\lstset{language=tamarin}

\paragraph{Round 1}
Client nonces (both $n_c$ and $n_k$) are generated as fresh terms.
Diffie-Hellman parameters ($p$ and $g$) are not modeled: instead, we use a single public constant \lstinline{'g'} that represents the generator. In Tamarin, this is needed to avoid having lots of partial deconstructions, a problem that causes an explosion in terms of complexity that usually leads to non-termination. As this public constant is known to everyone (including the attacker) there is no need to send it to the client in \nth{4} message. This simplification makes particularly sense if we consider that MTProto2.0 uses only six values for the generator $g$ (2, 3, 4, 5, 6 or 7).

Moreover, the proof-of-work is not used as it serves no real use for the protocol security. Finally, in our model we assume that the client has a way to get the public key of the server from its fingerprint. In Telegram, these keys are usually embedded in the application itself, resulting in the possibility of tampering \cite{MTProto2-Proverif}. Keypairs are generated using the following rule:

\begin{lstlisting}
rule RegisterPublicKey:
  let
    pkey        = pk(~skey)
    fingerprint = fpk(pkey)
  in
    [ Fr(~skey) ]
  -->
    [ !PrivateKey($X, ~skey), !PublicKey($X, pkey, fingerprint), Out(pkey) ]
\end{lstlisting}

In particular, in the model is the server who decides which public key to use and a single key fingerprint is sent to the client. As the attacker cannot add its own keys this should not matter.

\paragraph{Round 2} Another simplification is seen in the \nth{3} message: as encryption is not malleable in the symbolic model there is no need to introduce the hash of the plaintext message along with the message itself. Notice that this hash was actually used as a Message Authentication Code (MAC), which was exploited to check data integrity after decryption. This is applied to every message from now on. Public key encryption is defined using the built-in \lstinline{asymmetric-encryption} equational theory.

Time is not modeled, following from the formalisation in Proverif, and the generic key derivation function $\kdf{}$ has been renamed to $\func{genKey}{}$. Lastly, a public constant \lstinline{'StoC'} (\lstinline{'CtoS'}) has been used to mark the message from the client to the server (from the server to the client) that contains the server (client) Diffie-Hellman half key. This is needed to avoid adding an incorrect reflection attack in which the server receives the message he has previously sent. In reality, the encryption actually contains some data that allows matching messages. Symmetric encryption is modeled using the built-in \lstinline{symmetric-encryption} equational theory.

\lstset{language=tamarin}
In the implementation we also make great usage of pattern matching. Besides, the use of pattern matching is encouraged by Tamarin's manual as it usually decreases partial deconstructions. We have made use of it to ensure that half keys of client and server are actually in the form of \lstinline{'g'^~x}. This trick improves verification times by a lot, while it leaves the man-in-the-middle attack still possible\footnote{The attacker only needs to use its own ephemeral key and send the corresponding half key}.

\paragraph{Round 3} No simplification is needed for the last round, except for removing the $retryID$, meaning that we assume that the exchange is always successful. Notice that this is consistent with the MTProto2.0 specification: the client needs to retry to send his half key only when the server finds a duplicated key hash, but in our model this never happens as client and server use fresh values as Diffie-Hellman ephemeral keys, assuming correctness of the protocol.

\subsection{Additional implementation notes}
Every encrypted exchange is tagged with a public constant \lstinline{'AUTH_X'} which should result in better efficiency.

The server's nonce in the model might also be fixed. This models a flawed server implementation or a server which is lacking randomness. This is done by adding the following rules:

\begin{lstlisting}
rule GenerateRandomServerNonce:
    [ Fr(~ns) ]
  -->
    [ NS(~ns) ]

rule GenerateFixedServerNonce:
    []
  -->
    [ NS('FIXED_NS') ]
\end{lstlisting}

The server then, in the multiset rewriting rule premises, uses the \lstinline{NS(ns)} fact.

Finally, many compromise rules are created:
\begin{itemize}
  \item Compromise of server long-term key (asymmetric private key)
  \item Compromise of client secret nonce $n_k$
  \item Compromise of server ephemeral key (DH exponent)
  \item Compromise of client ephemeral key (DH exponent)
  \item Compromise of authorization keys
\end{itemize}

As compromise rules are very simple and similar to each other, we will only show an example:

\begin{lstlisting}
/* Reveals the client's DH secret exponent. */
rule CompromiseAuthProtClientExponent:
    [ !AuthProtClientEphemeralSecrets(nk, c) ]
  --[ CompromisedClientExponent(c) ]->
    [ Out(c) ]
\end{lstlisting}

Using these compromise rule, we can also check if the protocol is secure in the eCK model \cite{eCK}. The eCK model assumes, in the case of an authenticated key exchange, two parties, each having a long-term and an ephemeral secret. Of this four pieces of information, the eCK model allows to reveal any subset of these that does not contain both long-term and ephemeral secrets.
In the case of MTProto2.0, the authorization protocol does not strictly respect the eCK model as the client has no long-term secret. Moreover, intuitively, we can already notice that the protocol is not secure in the eCK model: revealing the client ephemeral secret $n_k$ allows the attacker to perform a classic man-in-the-middle attack on the Diffie-Hellman exchange \cite{MITM-DH}.

\subsection{Security properties verification}
Every rule in the protocol execution is labelled with action facts. We then use these to model security properties. Following the Proverif formalisation, we have modeled several forms of key agreement, authentication of parties and key secrecy, along with observational equivalence queries on the secret nonce $n_k$ and on the authorization key. In the next paragraphs we are going to examine lemmas and related results in further details.

\paragraph{Key agreement}
The following lemma models key agreement:
\begin{lstlisting}[numbers=left]
lemma LemmaAuthProtAgreement:
  "
    /* Whenever a client and a server negotiate an authorization key */
    ∀ nc ns nk authKey1 authKey2 #i #j.
      (
        ServerAcceptsAuthKey(nc, ns, nk, authKey1) @i ∧
        ClientAcceptsAuthKey(nc, ns, nk, authKey2) @j ∧

        /* and no secret was leaked */
        ¬(∃ sk #r.   CompromisedAuthKey(sk) @r) ∧
        ¬(∃ skey #r. CompromisedPrivateKey(skey) @r) ∧
        ¬(∃ n #r.    CompromisedNk(n) @r) ∧
        ¬(∃ c #r.    CompromisedClientExponent(c) @r) ∧
        ¬(∃ s #r.    CompromisedServerExponent(s) @r)
      )
      ==>
      (
        /* then the authorization key is the same */
        ( authKey1 = authKey2 ) ∨
        
        /* 
         * or the server is actually running two different instances
         * of the protocol with the client
         */
        (
          ∃ #n1 #n2.
            ServerGeneratesNonce(ns) @n1 ∧
            ServerGeneratesNonce(ns) @n2 ∧
            ¬(#n1 = #n2)
        )
      )
  "
\end{lstlisting}

By commenting any line between 10-14 (inclusive) we can model agreement in presence of some information leakage.
Key agreement without leaks holds. It may be interesting to notice that key agreement holds even when both ephemeral keys are revealed as the attacker cannot force client and server to compute different keys. Compromising one at a time either server's private key or client nonce allows the attacker to execute a man-in-the-middle attack on the Diffie-Hellman exchange.

We can also prove a similar property: if client and server end a run of the protocol negotiating the same key in their unrelated sessions, then these sessions actually coincide.

\paragraph{Authentication}
Client authentication in the protocol obviously does not hold because the client does not authenticate itself and the server is willing to execute the protocol with anybody (including the attacker). This query captures this:

\begin{lstlisting}
lemma LemmaAuthProtAuthClientToServer:
  "
    ∀ nc ns nk authKey #i #j.
      /* Whenever a client has started a session with nonce nc */
      ClientStartsSession(nc) @i ∧

      /* and the server has sent an ACK for the session <nc, ns> */
      ServerSendsAck(nc, ns, nk, authKey) @j ∧

      /* and no secret was leaked */          
      ¬(∃ sk #r.   CompromisedAuthKey(sk) @r) ∧
      ¬(∃ skey #r. CompromisedPrivateKey(skey) @r) ∧
      ¬(∃ n #r.    CompromisedNk(n) @r) ∧
      ¬(∃ c #r.    CompromisedClientExponent(c) @r) ∧
      ¬(∃ s #r.    CompromisedServerExponent(s) @r)
      ==>
      (
        /* then a client has shared an authKey with the server */
        (
          ∃ #k.
          ClientAcceptsAuthKey(nc, ns, nk, authKey) @k
        ) ∨

        /* 
         * or the server is actually running two different instances
         * of the protocol with the client
         */
        (
          ∃ #k #l.
            ServerGeneratesNonce(ns) @k ∧
            ServerGeneratesNonce(ns) @l ∧
            ¬(#k = #l)
        )
      )
  "
\end{lstlisting}

However, we can prove that the server knows for sure that the client that negotiated the authorization key is the same who sent the third message. For the sake of brevity, we will not report the related lemma.

As anyone can create an authorization key with the server, this lack of authentication is not an issue. Instead, server authentication is fundamental and it is captured by the following lemma:
\begin{lstlisting}[numbers=left]
  lemma LemmaAuthProtAuthServerToClient:
    "
      ∀ nc ns nk authKey #i1.
        /* Whenever a client receives an ACK from the server */
        ClientReceivesAck(nc, ns, nk, authKey) @i1 ∧
        
        /* and no secret was leaked */
        ¬(∃ sk #r.   CompromisedAuthKey(sk) @r) ∧
        ¬(∃ skey #r. CompromisedPrivateKey(skey) @r) ∧
        ¬(∃ n #r.    CompromisedNk(n) @r) ∧
        ¬(∃ c #r.    CompromisedClientExponent(c) @r) ∧
        ¬(∃ s #r.    CompromisedServerExponent(s) @r)
        ==>
        (
          /* then there is a session matching it on the server */
          ( 
            ∃ #j.
            ServerAcceptsAuthKey(nc, ns, nk, authKey) @j ∧
            (∀ #i2. ClientReceivesAck(nc, ns, nk, authKey) @i2 ==> #i1 = #i2)
          ) ∨

          /* or the server has reused the same nonce */
          (
            ∃ #j1 #j2.
              ServerGeneratesNonce(ns) @j1 ∧
              ServerGeneratesNonce(ns) @j2 ∧
              ¬(#j1 = #j2)
          )
        )
    "
\end{lstlisting}

This lemma holds, meaning that the server is authenticated to the client. Notice that line 19 is used to model injectivity.

\paragraph{Key secrecy}
Another fundamental property of the authentication protocol is key secrecy: when a client and a server negotiate a key, they must be sure that the key is known only to them. This property is formalised with the following lemma:
\begin{lstlisting}
lemma LemmaAuthProtKeySecrecy:
  "
    /* Whenever client and server negotiated a key */
    ∀ nc ns nk authKey #i #j #k.
      ClientAcceptsAuthKey(nc, ns, nk, authKey) @i ∧
      ServerAcceptsAuthKey(nc, nk, nk, authKey) @j ∧

      /* and the attacker knows it */
      K(authKey) @k
      ==>
      /* then some secret was leaked */
      (
        (∃ #r.      CompromisedAuthKey(authKey) @r) ∨
        (∃ skey #r. CompromisedPrivateKey(skey) @r) ∨
        (∃ #r.      CompromisedNk(nk) @r) ∨
        (∃ c #r.    CompromisedClientExponent(c) @r) ∨
        (∃ s #r.    CompromisedServerExponent(s) @r)
      )

  "
\end{lstlisting}

\paragraph{Observational equivalence}
Attempts to prove observational equivalence for client's secret nonce $n_k$ and authorization key, unfortunately, do not terminate. Notice that observational equivalence is often resources-consuming.

Observational equivalence in Tamarin is expressed using the \lstinline{diff/2} operator, which, basically, duplicates every rule, one for the left-hand side and one for the right-hand side of the operator, and tries to find a difference between the two traces.

In the formalisation, we created the two following rules:

\begin{lstlisting}
/*
 * The secret nonce nk generated by the client is indistinguishable 
 * from a fresh value.
 */
rule RuleAuthProtNkEquivalence:
    [
      !AuthProtClientEphemeralSecrets(nk, b),
      Fr(~n)
    ]
  -->
    [ Out(diff(nk, ~n)) ]

/*
 * A negotiated authorization key authKey is indistinguishable from a 
 * fresh value.
 */
rule LemmaAuthProtAuthKeyEquivalence:
    [
      !AuthKeyClient(server, authKey),
      Fr(~freshAuthKey)
    ]
  -->
    [ Out(diff(~freshAuthKey, authKey)) ]
\end{lstlisting}





\section{Cloud chat encryption schema}
\label{sec:cloud-chat-formalisation}

\subsection{Encryption formalisation}
Cloud chat encryption has been simplified to suit the symbolic model better.
First of all, the key derivation function returns only the key. Notice that, as both key and initialization vector are derived from the same terms, once the adversary is able to compute the key it would be able to compute the IV as well. Hence, it is not modeled as it would only add complexity to the model without bringing any benefit.

A function \lstinline{msgKey/2} is used to create the message key from the message and the authorization key. Then, the message key is used to create the encryption key for the message, together with the authorization key, using the \lstinline{genKey/2} primitive. The plaintext message is encrypted using the built-in \lstinline{symmetric-encryption}. The final message that is sent on the channel is composed of the fingerprint of the authorization key (using \lstinline{keyID/1}), the message key and the encrypted message. Notice that functions \lstinline{msgKey/2}, \lstinline{genKey/2} and \lstinline{keyID/1} have no associated equation (i.e. are considered perfect hashing functions).

Four different rules have been created: two are used to exchange messages from client to server and the other two for messages from server to client. This approach allows us to test for secrecy in both directions. Here is an example of the rule used to model a client to server message:

\begin{lstlisting}
  rule ClientCloudChatSendsMessage:
    let
      msg = <'CC_CtoS', ~sessionID, ~m>
      mk  = msgKey(msg, authKey)
      key = genKey(mk, authKey)
      c   = <keyID(authKey), mk, senc(msg, key)>
    in
      [
        !AuthKeyClient($Server, authKey),
        Fr(~sessionID),
        Fr(~m)
      ]
    --[ ClientSendsCloudMessage(~sessionID, ~m, authKey) ]->
      [ Out(c) ]
\end{lstlisting}

Notice on line 3 that we use a public constant \lstinline{'CC_CtoS'} to differentiate client to server from server to client messages.

\subsection{Security properties verification}

\paragraph{Secrecy and forward secrecy} The cloud chat encryption schema is essentially employed to obtain secrecy on messages exchanges between a client and a server, after they have negotiated an authorization key.
The following lemma proves secrecy from client to server. A very similar one is used to verify secrecy of messages from server to client.

\begin{lstlisting}
lemma LemmaCloudChatSecrecyClientToServer:
  "
    ∀ sid msg authKey #i #j #r.
      (
        /* Whenever a client sends a cloud message to the server */
        ClientSendsCloudMessage(sid, msg, authKey) @i ∧

        /* and the server receives it */
        ServerReceivesCloudMessage(sid, msg, authKey) @j ∧

        /* and the attacker knows it */
        K(msg) @r
      )
      ==>
      (
        /* then some secret was compromised */
        (∃ #r.      CompromisedAuthKey(authKey) @r) ∨
        (∃ skey #r. CompromisedPrivateKey(skey) @r) ∨
        (∃ n #r.    CompromisedNk(n) @r) ∨
        ((∃ c #r.   CompromisedClientExponent(c) @r) ∧
        (∃ s #r.    CompromisedServerExponent(s) @r))
      )
  "
\end{lstlisting}

This lemma actually means that messages are secure, unless:
\begin{itemize}
  \item The private key of the server is compromised
  \item The secret nonce $n_k$ of the client is compromised
  \item Diffie-Hellman exponents of client or server
\end{itemize}

As seen in \cref{sec:auth-prot-formalisation}, compromising any of these secrets leads to a lack of secrecy on they key. Additionally, also che compromise key itself can be compromised to break cloud chat secrecy.

As the lemma above is also strict (i.e. removing any event leads to an attack trace), this also means that perfect forward secrecy does not hold in cloud chats as an attacker that is able to compromise the authorization key can decrypt both past and future messages (as well as forging them, effectively impersonating the client to the server).

\section{Secret chat}
A secret chat can be initiated after clients have both shared an authorization key with the server. To create a shared secret (the \textit{secret chat key}) between two clients, MTProto2.0 uses, as already seen in \cref{sec:secret-chat}, a Diffie-Hellman key exchange in which the Telegram server acts as a forwarder.

As we have already analyzed security of the authorization protocol and of the cloud chat encryption in \cref{sec:auth-prot-formalisation,sec:cloud-chat-formalisation}, in the formalisation we do not encrypt messages with the authorization key, nor we use the server as a forwarder. Instead, we execute the secret chat protocol exchanges as if they were plaintext: this allows the attacker to act as the server (i.e. having knowledge of both authorization key and forwarding messages) and to manipulate messages freely \cite{MTProto2-Proverif}. Moreover, removing this layer of encryption allows us to simplify the protocol model, to formulate stronger security properties and to obtain higher efficiency.

\subsection{Exchanges formalisation}
%% Secret-chat protocol %%
\begin{figure}[!t]
  \setlength{\instdist}{3cm}
  \setmscoptions
  \begin{msc}{}
    \setmscscale{.8}

    \declinst{alice}{}{Alice}
    \declinst{bob}{}{Bob}

    \action*{\parbox{3.5cm}{\centering
        Knows $\key{AS}$\\
        Knows $g$\\
        Generates $sID$\\
        a $\in \group{p}$\\
        $g_a := \modexp{g}{a}{p}$
      }}{alice}
    \action*{Knows $\key{BS}$}{bob}

    \nextlevel[7]
    \mess{$sID, A, B, g_a$}{alice}{bob}
    \nextlevel

    \action*{\parbox{3.5cm}{\centering
        Knows $g$\\
        $b \in \group{p}$\\
        $g_b := \modexp{g}{b}{p}$\\
        $\key{AB} := \modexp{g_a}{b}{p}$
      }}{bob}

    \nextlevel[6]
    \mess{$sID, A, B, g_b, \func{fps}{k_{AB}}$}{bob}{alice}
    \nextlevel

    \action*{\parbox{4cm}{\centering
        $\key{AB} := \modexp{g_b}{a}{p}$\\
        Generates $msg$
      }}{alice}

    \nextlevel[3]
    \condition{Out-of-band $\key{AB}$ fingerprint comparison}{alice,bob}
    \nextlevel[4]
    \mess{$\enc{msg}{\key{AB}}$}{alice}{bob}

  \end{msc}

  \centering
  \caption{MTProto2.0 Secret-chat protocol formalisation}
  \label{fig:secret-chat-protocol-formalisation}
\end{figure}

First of all, as we consider authorization keys inherently secure, we separate this model from the authorization one. Instead, we replace the creation of authorization keys with a simple rule:

\begin{lstlisting}
rule GenerateAuthKey:
    [ In(user), Fr(~a), Fr(~b) ]
  --[ ChoosePrincipal(user, 'Alice', 'Bob') ]->
    [ !AuthKeyClient(user, $Server, 'g' ^ ~a ^ ~b) ]
\end{lstlisting}

Skipping the authorization protocol setup makes this model much more efficient.

Additionally, we allow the attacker to register an authorization key for itself:

\begin{lstlisting}
rule GenerateAttackerAuthKey:
  let authKey = 'g' ^ ~a ^ ~b in
    [ Fr(~a), Fr(~b) ]
  -->
    [ !AuthKeyClient('Eve', $Server, authKey), Out(authKey) ]
\end{lstlisting}


Following the Proverif formalisation, we explicitly name clients and we also allow the attacker to choose their roles (i.e. who is the initiator, who the initiator talks to, who the responder is). We apply restriction rules to make sure that the initiator is always honest: % TODO: i've never said what restriction rules were... this maybe needs to be added?

\begin{lstlisting}
restriction RestrictionChoosePrincipal:
  "
    ∀ X Y Z #i.
      ChoosePrincipal(X, Y, Z) @i ==> ((X = Y) ∨ (X = Z))
  "
\end{lstlisting}

Having explicitly named clients, we will be able to prove additional authentication security properties.
Finally, we model the QR-code comparison with the following rules:

\begin{lstlisting}
rule PerformOutOfBandKeyComparison:
    [ 
      !QR(aID, aUser, bUser, sessionKey),
      !QR(bID, bUser, aUser, sessionKey)
    ]
  --[ 
      /* Rule out the possibility of sessions of a client with itself */
      NotEq(aUser, bUser),

      OutOfBandKeyComparisonSucceeded(aID, aUser, bUser, sessionKey),
      OutOfBandKeyComparisonSucceeded(bID, bUser, aUser, sessionKey)
    ]->
    [
      !QROK(aID, aUser, bUser, sessionKey),
      !QROK(aID, bUser, aUser, sessionKey)
    ]

rule SkipOutOfBandKeyComparison:
    [ !QR(aID, aUser, bUser, sessionKey) ]
  --[ OutOfBandKeyComparisonSkipped(aID, aUser, sessionKey)]->
    [ !QROK(aID, aUser, bUser, sessionKey) ]

restriction RestrictionNotEqual:
  "∀ x y #i. NotEq(x, y) @i ==> ¬(x = y)"
\end{lstlisting}

The persistent fact \lstinline{!QR} is created by each party after they have computed the secret chat key. The above rules model both the correct and incorrect behaviour of clients and allows to specify properties in which a specific behaviour is executed. The correct behaviour is the one express by the \lstinline{PerformOutOfBandKeyComparison} rule, where the client implicitly checks that the sessionKey he computed is the same of the other party. The incorrect behaviour in rule \lstinline{SkipOutOfBandKeyComparison} models a client that does not match the key fingerprint: as this is the only way of authenticating parties to each other, skipping the comparison possibly results in a Diffie-Hellman man-in-the-middle attack.

Additionally, two rules have been added to allow message sending and receiving:

\begin{lstlisting}
rule SecretChatSend:
  let
    mk  = msgKey(~m, sessionKey)
    key = genKey(mk, sessionKey)
    c   = <keyID(sessionKey), mk, senc(~m, key)>
  in
    [ 
      !SecretChatClient(X, iUser, rUser, xID, chatID, sessionKey, authKey),
      Fr(~m)
    ]
  --[
      ClientSendsSecretChatMsg(chatID, X, iUser, rUser, sessionKey, ~m)
    ]->
    [ Out(c) ]

rule SecretChatReceive:
  let
    mk  = msgKey(~m, sessionKey)
    key = genKey(mk, sessionKey)
    c   = <keyID(sessionKey), mk, senc(~m, key)>
  in
    [
      !SecretChatClient(X, iUser, rUser, xID, chatID, sessionKey, authKey),
      In(c)
    ]
  --[ ClientReceivesSecretChatMsg(chatID, X, iUser, rUser, sessionKey, ~m) ]->
    []
\end{lstlisting}

\Cref{fig:secret-chat-protocol-formalisation} shows the secret chat protocol, with the above simplification.

\subsection{Security properties verification}

Two main properties should be satisfied by the secret chat protocol: secrecy and authentication.


\paragraph{Secrecy} This is, of course, one of the most important property for an end-to-end chat. MTProto2.0 guarantees secrecy under the assumption that clients compare the secret chat key fingerprint out-of-band. This mechanism is also used, for example, by Signal \cite{SignalProtocol}. Formally, Tamarin is able to prove the following secrecy lemma:

\begin{lstlisting}
lemma LemmaSecretChatSecrecy:
  "
    /* Whenever the client sends a secret chat message msg */
    ∀ chatID X iUser rUser sessionKey msg #i #j.
      ClientSendsSecretChatMsg(chatID, X, iUser, rUser, sessionKey, msg) @i ∧
      
      /* but the attacker knows it */
      K(msg) @j
      ==>
      (
        /* then clients skipped the QR validation */
        ∃ a #r. OutOfBandKeyComparisonSkipped(a, X, sessionKey) @r
      )
  "
\end{lstlisting}

In another words, whenever a client sends a message then it is secret, unless clients did not skip the key fingerprint comparison. Let us stress on the fact that this result is independent from the secrecy of the authorization key. Hence, if clients compare the fingerprint correctly, end-to-end encryption holds even against a compromised server.

\paragraph{Authentication}
Many variants of authentication have been proven. The most basic one is the following:

\begin{lstlisting}
  lemma LemmaSecretChatAuthentication1:
  "
    /* Whenever a client received a secret chat message */
    ∀ chatID1 X iUser rUser msg sessionKey #i.
      ClientReceivesSecretChatMsg(chatID1, X, iUser, rUser, sessionKey, msg) @i
      ==>
      (
        /* then it was sent by another (honest) client */
        (∃ Y chatID2 #r. 
          ClientSendsSecretChatMsg(chatID2, Y, iUser, rUser, sessionKey, msg) @r) ∨
        (∃ Y chatID2 #r. 
          ClientSendsSecretChatMsg(chatID2, Y, rUser, iUser, sessionKey, msg) @r) ∨

        /* or clients involved have skipped the QR validation */
        (
          ∃ Y xID yID #r1 #r2. 
            OutOfBandKeyComparisonSkipped(xID, X, sessionKey) @r1 ∧
            OutOfBandKeyComparisonSkipped(yID, Y, sessionKey) @r2
        )
      )
  "
\end{lstlisting}

This lemma states that if an honest client has received a message in session $chatID1$ and with key $sessionKey$, then there was another honest client which sent it in session $chatID2$ with the same key or that clients have skipped verification. Notice that we cannot match session (i.e. $chatID1 = chatID2$) as the server is able to forward messages modifying the session identifier. As also pointed to by M. Miculan and N. Vitacolonna \cite{MTProto2-Proverif}, this does not seem to pose any security risks.

As already stated, skipping validation of the key's fingerprint results in a classical Diffie-Hellman MitM attack, which obviously invalidates authentication.

\paragraph{Observational equivalence} We have tried to model several observational equivalence properties, such as:
\begin{itemize}
  \item Indistinguishability of a client's exponent from a random value. This should not hold as the attacker can compute the DH key (using the exponent and the other half of the key) and compare fingerprints.
  \item Indistinguishability of a secret chat key and a random value. This should not hold as the attacker can compare the key fingerprint with the \nth{2} message of the protocol.
  \item Indistinguishability of message exchanged with the secret chat key and a random value. This should hold, as long as the key used to encrypt the message is kept secret.
\end{itemize}

Again, Tamarin does not terminate when trying to prove these properties. These three rules are used to model observational equivalence:

\begin{lstlisting}
rule SecretChatExponentRoR:
    [
      !SecretChatClient(X, iUser, rUser, xID, chatID, sessionKey, authKey),
      Fr(~z)
    ]
  -->
    [ Out(diff(~z, xID)) ]

rule SecretChatSessionKeyRoR:
    [ 
      !SecretChatClient(X, iUser, rUser, xID, chatID, sessionKey, authKey),
      Fr(~r)
    ]
  -->
    [ Out(diff(sessionKey, ~r)) ]

rule SecretChatReceiveRoR:
  let
    mk  = msgKey(sessionKey, ~m)
    key = kdfKey(sessionKey, mk)
    c   = <keyID(sessionKey), mk, senc(~m, key)>
  in
    [ 
      !SecretChatClient(X, iUser, rUser, xID, chatID, sessionKey, authKey),
      In(c), Fr(~r)
    ]
  -->
    [ Out(diff(~m, ~r)) ]
\end{lstlisting}


\chapter{Comparison with Proverif}
\input{./chapters/06_comparison}

%% Fine dei capitoli normali, inizio dei capitoli-appendice (opzionali)
% \appendix

%% Parte conclusiva del documento; tipicamente per riassunto, bibliografia e/o indice analitico.
\backmatter

%% Riassunto (opzionale)
%\summary

%% Bibliografia (praticamente obbligatoria)
\bibliographystyle{plain_\languagename}
\bibliography{thud}

\end{document}